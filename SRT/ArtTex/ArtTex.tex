% MIT License

% Copyright (c) 2022 Chiyuru

% Permission is hereby granted, free of charge, to any person obtaining a copy of this software and associated documentation files (the "Software"), 
% to deal in the Software without restriction, including without limitation the rights
% to use, copy, modify, merge, publish, distribute, sublicense, and/or sell
% copies of the Software, and to permit persons to whom the Software is
% furnished to do so, subject to the following conditions:

% The above copyright notice and this permission notice shall be included in all copies or substantial portions of the Software.

% THE SOFTWARE IS PROVIDED "AS IS", WITHOUT WARRANTY OF ANY KIND, EXPRESS OR IMPLIED, INCLUDING BUT NOT LIMITED TO THE WARRANTIES OF MERCHANTABILITY,
% FITNESS FOR A PARTICULAR PURPOSE AND NONINFRINGEMENT. IN NO EVENT SHALL THE AUTHORS OR COPYRIGHT HOLDERS BE LIABLE FOR ANY CLAIM, DAMAGES OR OTHER
% LIABILITY, WHETHER IN AN ACTION OF CONTRACT, TORT OR OTHERWISE, ARISING FROM,
% OUT OF OR IN CONNECTION WITH THE SOFTWARE OR THE USE OR OTHER DEALINGS IN THE SOFTWARE.

% 宏定义一些数学符号

\def\f#1#2{\frac{#1}{#2}}
\def\d#1{\dot{#1}}
\def\dd#1{\ddot{#1}}
\def\fd#1#2{\frac{d #1}{d #2}}
\def\fp#1#2{\frac{\partial #1}{\partial #2}}
\def\b#1{\boldsymbol{#1}}


\documentclass[UTF8]{ctexart}

\usepackage{tikz}
\usetikzlibrary{3d,quotes,angles}
\usepackage{amsmath}
\usepackage{cases}
\usepackage{cite}
\usepackage{graphicx}
\usepackage[margin=1in]{geometry}
\geometry{a4paper}
\usepackage{fancyhdr}
\pagestyle{fancy}
\fancyhf{}

\title{SRT创新计划专项结题报告\\倒立摆控制}
\author{刘锦坤}
\date{\today}
\pagenumbering{arabic}

\begin{document}

%\fancyhead[L]{驰雨Chiyuru}
\fancyhead[C]{倒立摆控制}
\fancyfoot[C]{\thepage}

\maketitle
\tableofcontents
\newpage

\section{摘要}

本文主要总结了本次倒立摆项目中所做的工作。

\section{PD控制参数的理论计算}

在本次SRT项目中,选用PD方法来完成对倒立摆的控制,可以建立动力学模型,对PD控制器的参数进行理论计算。

\subsection{动力学方程}

系统动力学模型示意图如图所示:

\begin{figure}[htbp]
    \centering
    \begin{tikzpicture}[x={(-1cm,-1cm)},y={(1.5cm,0cm)},z={(0cm,1.5cm)}]
        % 定义坐标轴
        \draw[->] (0,0,0) -- (4,0,0) node[below] {$X$};
        \draw[->] (0,0,0) -- (0,4,0) node[right] {$Y$};
        \draw[->] (0,0,0) -- (0,0,4) node[above] {$Z,z$};
        % 画水平杆子
        \draw[line width=3pt] (0,0,0) -- (3,3,0) node[below] {$m_1,l_1$};
        % 画竖直虚线
        \draw[dashed] (3,3,0) -- (3,3,4);
        % 画竖直杆子
        \draw[line width=3pt] (3,3,0) -- (2,5,4) node[above] {$m_2,l_2$};
        % 两杆铰接处
        \draw[fill=black] (3,3,0) circle (0.05);
        % 标记phi角和tau
        \draw[thick,->] (0.5,0,0) arc (0:40:0.5 and 0.5);
        \node[font=\LARGE] at (0.9,0.4,0) {$\phi,\tau$};
        % 标记theta角
        \draw[thick,->] (3,3,1) arc (0:35:-1.0 and 1.0);
        \node[font=\LARGE] at (3,3.3,1.2) {$^\theta$};
        % 随杆坐标轴
        \draw[->] (0,0,0) -- (4.3,4.3,0) node[below] {$x$};
        \draw[->] (0,0,0) -- (-2.5,2.5,0) node[above] {$y$};
        \draw[->] (0,0,0) -- (0,0,4);
    \end{tikzpicture}
    \caption{倒立摆示意图}
\end{figure}

$XYZ$坐标系为固定坐标系,而$xyz$为随着水平杆转动的坐标系。记水平杆的质量$m_1$,长度$l_1$,树枝干的质量$m_2$,长度$l_2$,水平杆的转动角度为$\phi$,竖直杆的转动角度$\theta$,假设两杆的质量都是均匀分布。电机的驱动力矩为$\tau$,重力加速度为$g$,下面从第二类Lagrange方程出发推倒倒立摆的动力学方程。

水平杆的动能为

\begin{equation}
    T_1 = \f 1 6 m_1 l_1^2 {\d{\phi}}^2
\end{equation}

竖直杆的动能为为质心动能和相对之心动能之和,其质心速度为:

\begin{equation}
    \vec v = -\f 1 2 l_2 \d{\phi} sin \theta \hat x
    +(\d{\phi} l_1+\f 1 2 l_2 \d{\theta} cos \theta )\hat y
    - \f 1 2 l_2 \d{\theta} sin \theta \hat z 
\end{equation}

其中$\hat{x},\hat{y},\hat{z}$为$xyz$坐标系各个方向的单位矢量,故其质心动能为

\begin{equation}
    \begin{aligned}
    T_{2c} &= \f 1 2 m_2 \vec v^2\\ 
    &= \f 1 2 m_2 {l_1}^2 {\d \phi}^2
    +\f 1 2 m_2 \d \phi \d \theta l_1 l_2 cos \theta
    +\f 1 8 m_2 {l_2}^2 {\d \theta}^2
    +\f 1 8 m_2 {l_2}^2 {\d \phi}^2 sin^2 \theta
    \end{aligned}
\end{equation}

而竖直杆的相对质心动能为转动贡献,即为

\begin{equation}
    T_{2r} = \f {1} {24} m_2 l_2^2 {\d \theta}^2+ \f {1} {24} m_2 l_2^2 {\d \phi}^2 sin^2 \theta
\end{equation}

系统势能为

\begin{equation}
    V = \f 1 2 m_2 g l_2 cos \theta
\end{equation}

系统的Lagrange量为

\begin{equation}
    \begin{aligned}
        L &= T - V = T_1 + T_{2c} + T_{2r} - V\\
        &= \f 1 6 m_1 l_1^2 {\d{\phi}}^2
        +\f 1 2 m_2 {l_1}^2 {\d \phi}^2
        +\f 1 2 m_2 \d \phi \d \theta l_1 l_2 cos \theta
        +\f 1 8 m_2 {l_2}^2 {\d \theta}^2\\
        &+\f 1 8 m_2 {l_2}^2 {\d \phi}^2 sin^2 \theta
        +\f {1} {24} m_2 l_2^2 {\d \theta}^2
        + \f {1} {24} m_2 l_2^2 {\d \phi}^2 sin^2 \theta
        -\f 1 2 m_2 g l_2 cos \theta
    \end{aligned}
\end{equation}

保留到二阶小量,可以得到:

\begin{equation}
    L = \f 1 6 m_1 l_1^2 {\d{\phi}}^2
    + \f 1 2 m_2 {l_1}^2 {\d \phi}^2
    + \f 1 2 m_2 \d \phi \d \theta l_1 l_2
    + \f 1 8 m_2 {l_2}^2 {\d \theta}^2
    + \f {1} {24} m_2 l_2^2 {\d \theta}^2
    + \f 1 4 m_2 g l_2\theta^2
\end{equation}

带入第二类Lagrange方程,得到

\begin{equation}
    \begin{cases}
        \f 1 3 m_1 l_1 \dd{\phi} + m_2 l_1 \dd{\phi} + \f 1 2 m_2 l_1 l_2 \dd{\theta} = \tau\\
        \f 1 2 m_2 l_1 \dd{\phi} + \f 1 3 m_2 l_2\dd\theta - \f 1 2 m_2 g \theta = 0
    \end{cases}
\end{equation}

记$\b{q}=[\phi , \theta]^T$,有

\begin{equation}
    \b{M} \b{\dd q} + \b{K} \b{q} = \b{\tau}
\end{equation}

其中

\begin{equation}
    \b M = \begin{bmatrix}
        \f 1 3 m_1 l_1 + m_2 l_1 & \f 1 2 m_2 l_1 l_2\\
        \f 1 2 m_2 l_1  & \f 1 3 m_2 l_2
    \end{bmatrix},
    \b K = \begin{bmatrix}
        0 & 0\\
        0 & \f 1 2 m_2 g
    \end{bmatrix},
    \b \tau = \begin{bmatrix}
        \tau\\
        0
    \end{bmatrix}
\end{equation}

这就是倒立摆的动力学方程。

\subsection{本征模块}

我们先分析系统的本征模块,设其一个本征模块为$\b q(t) = \b \eta e^{\lambda t}$,代入方程$\b{M} \b{\dd q} + \b{K} \b{q} = 0$得到特征方程:

\begin{equation}
        \det(\lambda^2\b M + \b K) = 0
\end{equation}

且对应的

\begin{equation}
    (\lambda^2\b M + \b K)
\end{equation}

\section{实验原理}

\subsection{xxx方程}
在xx,xxx,xxxx条件下,考察条件为xx的xx的情况,利用xxxx定律在无位移的水平方向和有位移的竖直方向分别列出以下方程:

\begin{numcases}{}
    T_2cos\alpha_2 - T_1cos\alpha_1 = 0 \\
    T_2sin\alpha_2 - T_1sin\alpha_1 = \rho dx\frac{\partial^2y }{\partial x^2} 
\end{numcases}

\subsection{xxx情况下的边界条件和xx现象}
xxxx时发生xxxx现象。由xxx方程可知,xxx波形为$y^+=f(vt+x)$,xxx波形为$y^-=f(vt-x)$。

\subsection{xx在xxx条件下的xxx现象}
Complete by yourself!


\section{实验过程与数据分析}
\subsection{A.在xx条件下测量xxx}
\subsubsection{$a1. $计算出xx的电阻和电感}
在xx上将xx的两端串联xx和xx相连,将xx的两端串联进xx,分别将xx接在$L_1$,$L_2$,xx的两端测量xx并记录。
\subsubsection{$a2. $Complete by yourself!}
Complete by yourself!
\subsubsection{$a3. $Complete by yourself!}
实验得到的数据如下:

\begin{center}
\begin{tabular}{|c|c|c|c|c|c|}
 \hline
线圈名称 & R'(Ω) & Va(V) & V(V) & Vr'(V) & Vo(V)\\
 \hline
线圈1(空气芯) & 123 & 456 & 789 & 012 & 345\\
 \hline
线圈2(空气芯) & 123 & 456 & 789 & 012 & 345\\
 \hline
线圈3(铝芯) & 123 & 456 & 789 & 012 & 345\\
 \hline
线圈4(铝芯) & 123 & 456 & 789 & 012 & 345\\
 \hline
\end{tabular}
\end{center}

\subsection{展示一下行间公式}
\subsubsection{行间公式}
% 行间公式用 $$ $$ 或者 \[ \] 来框住都可以,但在 LaTeX 中前者会改变行文的默认行间距,因此不推荐采用。
\paragraph{}这是一个不确定度计算。
\[
U_k=tinv(x,y)×s_k=xxx
\]
\subsubsection{相对于行内公式}
这是一个不确定度计算:$U_k=tinv(x,y)×s_k=xxx$


\section{分析与讨论}

\subsection{误差分析}

\subsubsection{实验中的系统误差}
来自xxx的精度影响。

受空间内xx与xx的干扰。

\subsubsection{实验中的偶然误差}
接线时可能有xxx情况,导致xxx。xx上的xx在某情况下有xx的问题存在,经反复调整后得以正常测量。

\subsection{实验后的思考}
可说明自己做本实验的总结、收获和体会,对实验中发现的问题提出自己的建议。

\newpage
%图一般很大,建议换页。
\section{原始数据}
\begin{center}
    Change the picture by yourself!
    
    
%    \includegraphics{picture/example.png}
\end{center}



\bibliographystyle{plain}
\bibliography{./template}  %bib文件名

\end{document}