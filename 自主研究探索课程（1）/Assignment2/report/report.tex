% MIT License

% Copyright (c) 2022 Chiyuru

% Permission is hereby granted, free of charge, to any person obtaining a copy of this software and associated documentation files (the "Software"), 
% to deal in the Software without restriction, including without limitation the rights
% to use, copy, modify, merge, publish, distribute, sublicense, and/or sell
% copies of the Software, and to permit persons to whom the Software is
% furnished to do so, subject to the following conditions:

% The above copyright notice and this permission notice shall be included in all copies or substantial portions of the Software.

% THE SOFTWARE IS PROVIDED "AS IS", WITHOUT WARRANTY OF ANY KIND, EXPRESS OR IMPLIED, INCLUDING BUT NOT LIMITED TO THE WARRANTIES OF MERCHANTABILITY,
% FITNESS FOR A PARTICULAR PURPOSE AND NONINFRINGEMENT. IN NO EVENT SHALL THE AUTHORS OR COPYRIGHT HOLDERS BE LIABLE FOR ANY CLAIM, DAMAGES OR OTHER
% LIABILITY, WHETHER IN AN ACTION OF CONTRACT, TORT OR OTHERWISE, ARISING FROM,
% OUT OF OR IN CONNECTION WITH THE SOFTWARE OR THE USE OR OTHER DEALINGS IN THE SOFTWARE.

% 宏定义数学公式中的粗体
\newcommand{\mb}[1]{\boldsymbol{#1}}

\documentclass[UTF8]{ctexart}

\usepackage{amsmath}
\usepackage{cases}
\usepackage{cite}
\usepackage{graphicx}
\usepackage[margin=1in]{geometry}
\geometry{a4paper}
\usepackage{fancyhdr}
\pagestyle{fancy}
\fancyhf{}

\title{孤立系统中单管引水的水轮机调节系统\\小波动特性分析报告}
\author{刘锦坤}
\date{}

\pagenumbering{arabic}

\begin{document}

\fancyhead[C]{波动特性分析报告}
\fancyfoot[C]{\thepage}

\maketitle
\tableofcontents

\newpage

\section{说明}

本次作业所有文件均已上传至附件,代码保存在code文件夹,图片保存在在pic文件夹,问题[n]对应的代码为Question[n].py文件,util.py文件为编写的工具函数文件,其中包括了若干矩阵运算函数。

\section{问题一}

运行Question1.py文件,即可打印出AR矩阵元素和转移矩阵的元素:

\begin{verbatim}
    Matrix A_R is:
[[ -0.223        0.076        0.           0.1665      -0.1       ]
 [-10.          -0.4        -10.           0.           0.        ]
 [ -8.          -0.32        -8.2          0.           0.        ]
 [ 13.26435742   0.57419531  13.37890625  -0.54986471  -0.05136719]
 [  0.           0.           0.           0.           0.        ]]
    The transition matrix is:
[[ 9.92406723e-01  2.95483905e-03  9.61190807e-04  6.31148589e-03
  -3.83722343e-03]
 [-3.27432401e-01  9.86337929e-01 -3.27586450e-01 -1.09813331e-03
   6.65067258e-04]
 [-2.60885797e-01 -1.08860649e-02  7.31328490e-01 -8.76193562e-04
   5.30655930e-04]
 [ 4.28617119e-01  1.95475459e-02  4.33162394e-01  9.80517570e-01
  -2.82955859e-03]
 [ 0.00000000e+00  0.00000000e+00  0.00000000e+00  0.00000000e+00
   1.00000000e+00]]
\end{verbatim}

同时绘制出转速$n$的过渡过程曲线如图1所示,可以看到,在经过一段时间后,转速$n$会进入一个新的稳定值。

\begin{figure}[htbp]
    \centering
    \includegraphics[width=0.8\textwidth]{pic/n-t.png}
    \caption{转速$n$的过渡过程曲线}
\end{figure}

\section{问题二}

运行Question2.py文件,即可得到不同的$T_d,b_t$的值对应的过渡过程曲线如图2所示:

\begin{figure}[htbp]
    \centering
    \includegraphics[width=0.8\textwidth]{pic/dif_td_bt.png}
    \caption{$T_d,b_t$的不同值对应的过渡过程曲线}
\end{figure}

可以看到,在$T_d$和$b_t$都较小时,系统在扰动后偏离平衡态,在扰动下不稳定。而$b_t$增大后,系统再次进入稳定域,说明$b_t$的增大有助于系统的稳定。同时还能发现,在一定范围内$T_d$和$b_t$的增大都能减少系统的超调,提高系统的动态品质。

\section{问题三}

运行Question3.py文件,即可得到不同的$T_a$的值对应的过渡过程曲线如图3所示:

\begin{figure}[htbp]
    \centering
    \includegraphics[width=0.8\textwidth]{pic/dif_ta.png}
    \caption{$T_a$的不同值对应的过渡过程曲线}
\end{figure}

可以看到,$T_a$的增大会使得系统的超调量增大,调节时间变长,因此在一定范围内$T_a$的增大会使得系统的动态品质变差,但是由于超调量的增大,系统的稳定性很可能会增强。

\section{问题四}

运行Question4.py文件,即可得到不同的$T_w$的值对应的过渡过程曲线如图4所示:

\begin{figure}[htbp]
    \centering
    \includegraphics[width=0.8\textwidth]{pic/dif_tw.png}
    \caption{$T_w$的不同值对应的过渡过程曲线}
\end{figure}

可以看到,更大的$T_w$会使得系统向下的振荡幅度更大,因此在一定范围内$T_w$的增大会使得系统的动态品质变差,也可能会使得系统的稳定性变差。

\section{问题五}

这里我们可以进行一个理论推导分析系统的稳定性,我们知道任意矩阵$\mb A$可以分解为:

\begin{equation}
  \mb A = \mb P^{-1} \mb J \mb P
\end{equation}

其中$\mb J$是$\mb A$的Jordan标准型,$\mb P$是一个可逆矩阵。
\end{document}

