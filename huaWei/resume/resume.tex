\documentclass{resume}
\usepackage{zh_CN-Adobefonts_external} 
\usepackage{linespacing_fix}
\usepackage{cite}
\usepackage{hyperref}
\hypersetup{
    colorlinks=true,
    linkcolor=cyan,
    filecolor=magenta,      
    urlcolor=blue,
}

\begin{document}
\pagenumbering{gobble}

%***"%"后面的所有内容是注释而非代码,不会输出到最后的PDF中
%***使用本模板,只需要参照输出的PDF,在本文档的相应位置做简单替换即可
%***修改之后,输出更新后的PDF,只需要点击Overleaf中的“Recompile”按钮即可

%在大括号内填写其他信息,最多填写4个,但是如果选择不填信息,
%那么大括号必须空着不写,而不能删除大括号。
%\otherInfo后面的四个大括号里的所有信息都会在一行输出
%如果想要写两行,那就用两次这个指令(\otherInfo{}{}{}{})即可


%***********个人信息**************
\MyName{刘锦坤}
\sepspace
\SimpleEntry{liujk0725@outlook.com}
\SimpleEntry{18973738468}
\SimpleEntry{清华大学——行建书院}
\SimpleEntry{理论与应用力学+能源与动力工程}

%************照片**************
%照片需要放到images文件夹下,名字必须是you.jpg,注意.jpg后缀(可以去resume.cls第101行处修改),如果不需要照片可以不添加此行命令
%0.15的意思是,照片的宽度是页面宽度的0.15倍,调整大小,避免遮挡文字
\yourphoto{0.14}

%***********教育背景**************
\section{学习经历}
%***第一个大括号里的内容向左对齐,第二个大括号里的内容向右对齐
%***\textbf{}括号里的字是粗体,\textit{}括号里的字是斜体
\datedsubsection{\textbf{湖南省长沙市第一中学},湖南省拔尖创新人才培养计划,\textit{高中}}{2018.11 - 2022.6}
\datedsubsection{\textbf{清华大学},行建书院,理论与应用力学+能源与动力工程双学位,\textit{本科在读}}{2022.9 - 至今}
\begin{itemize}
  \item 2018年8月,入选长沙市一中拔尖创新人才培养计划,\\2018年11月进入长沙市一中,提前开始物理竞赛和高中知识学习。
  \item 2022年6月,经由强基计划,进入清华大学行建书院,\\选修理论与应用力学+能源与动力工程双学位。
\end{itemize}

%***********本科阶段成绩**************
\section{在校成绩}
截止到2024年6月:

\qquad \qquad \textbf{GPA}:3.922/4.0 

\textbf{选修课程:}
  
\qquad \qquad 数学分析,高等代数,动力学与控制基础,程序设计基础,数据结构,人工智能导论等共74学分。

%***********过往经历**************
\section{项目经验}
\datedsubsection{\textbf{本科生科研SRT项目}}{2023.3 - 2024.3}
\Content
{针对xxxx,实时检测xxxx情况(主要负责人)。}
{研究xxx算法、xxxx算法以及两者的结合。}
{完成xxxx与xxx的结合,在xxxx,目前已上线使用。}

\datedsubsection{\textbf{机器狗足球比赛},清华大学自动化系(在做)}{2020.1 - 2020.5}
\Content
{负责xxx项目,对xxx进行智能诊断。}
{搭建xxx网站,结合xxx,利用xxx,基于xxx实现xxx。}
{1.完成xxx;2.基于xxx,算法准确率在85\%左右。}

\datedsubsection{\textbf{基于机器学习的股票策略},人工智能导论课程项目(在做)}{2018.3 - 至今}
\Content
{针对xxx的不足,研究一种基于xxx对xxx进行xxx的方法。}
{基于xxx建立xxx模型,使xxx,基于xxx,设计一种xxx。}
{1.建立了xxx,并在xxx得到了验证;2.对xxx,算法准确率在95\%以上;3.对xxx可以自主学习,从而xxx。}

\datedsubsection{\textbf{大学生创新创业项目},xxx}{2016.3 - 2017.3}
\Content
{针对xxx,基于xxx,加入xx实现了xxx的功能(主要负责人)。}
{搭建xxx,利用xxx的输出,实现对xxx的控制。}
{1.实现了xxx;2.获得了xxx奖等荣誉。}

\section{专业技能}
\datedsubsection{\textbf{计算机方面}:xxx}{}
\datedsubsection{\textbf{算法方面}:xxx}{}
\datedsubsection{\textbf{英语方面}:xxx}{}
\sepspace

\section{奖励荣誉}
\datedsubsection{\textbf{比赛方面}:}{}
\datedsubsection{xxx}{2020}
\datedsubsection{xxx}{2020}
\datedsubsection{xxx}{2017}
\datedsubsection{\textbf{论文方面}:xxx}{2020}
\datedsubsection{\textbf{运动方面}:xxx}{2018、2019}

\end{document}