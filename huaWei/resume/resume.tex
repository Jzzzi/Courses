\documentclass{resume}
\usepackage{zh_CN-Adobefonts_external} 
\usepackage{linespacing_fix}
\usepackage{cite}
\usepackage{hyperref}
\hypersetup{
    colorlinks=true,
    linkcolor=cyan,
    filecolor=magenta,      
    urlcolor=blue,
}

\begin{document}
\pagenumbering{gobble}

%***"%"后面的所有内容是注释而非代码,不会输出到最后的PDF中
%***使用本模板,只需要参照输出的PDF,在本文档的相应位置做简单替换即可
%***修改之后,输出更新后的PDF,只需要点击Overleaf中的“Recompile”按钮即可

%在大括号内填写其他信息,最多填写4个,但是如果选择不填信息,
%那么大括号必须空着不写,而不能删除大括号。
%\otherInfo后面的四个大括号里的所有信息都会在一行输出
%如果想要写两行,那就用两次这个指令(\otherInfo{}{}{}{})即可


%***********个人信息**************
\MyName{刘锦坤}
\sepspace
\SimpleEntry{liujk0725@outlook.com}
\SimpleEntry{18973738468}
\SimpleEntry{清华大学——行建书院}
\SimpleEntry{理论与应用力学+能源与动力工程}

%************照片**************
%照片需要放到images文件夹下,名字必须是you.jpg,注意.jpg后缀(可以去resume.cls第101行处修改),如果不需要照片可以不添加此行命令
%0.15的意思是,照片的宽度是页面宽度的0.15倍,调整大小,避免遮挡文字
\yourphoto{0.14}

%***********教育背景**************
\section{学习经历}
%***第一个大括号里的内容向左对齐,第二个大括号里的内容向右对齐
%***\textbf{}括号里的字是粗体,\textit{}括号里的字是斜体
\datedsubsection{\textbf{湖南省长沙市第一中学},湖南省拔尖创新人才培养计划,\textit{高中}}{2018.11 - 2022.6}
\datedsubsection{\textbf{清华大学},行建书院,理论与应用力学+能源与动力工程双学位,\textit{本科在读}}{2022.9 - 至今}
\begin{itemize}
  \item 2018年8月,入选长沙市一中拔尖创新人才培养计划,\\2018年11月进入长沙市一中,提前开始物理竞赛和高中知识学习。
  \item 2022年6月,经由强基计划,进入清华大学行建书院,\\选修理论与应用力学+能源与动力工程双学位。
\end{itemize}

%***********本科阶段成绩**************
\section{在校成绩}
截止到2024年6月:

\qquad \qquad \textbf{GPA}:3.922/4.0 

\sepspace

\textbf{选修课程:}
  
\qquad \qquad 数学分析,高等代数,动力学与控制基础,程序设计基础,数据结构,人工智能导论等。

%***********过往经历**************
\section{项目经验}
\datedsubsection{\textbf{本科生科研SRT项目}}{2023.3 - 2024.3}
\Content
{基于Matlab的倒立摆智能控制算法设计与应用}
{结合力学模型,提出稳定的控制方法。}
{分析了倒立摆的理论力学模型,对PID参数进行计算分析,在Simulink中实现了对倒立摆的平衡控制,并实现了多摆的协同平衡控制。}

\datedsubsection{\textbf{机器狗足球比赛},清华大学自动化系(在做)}{2024.4 - 2024.6(预期)}
\Content
{基于小米提供的CyberDog,设计算法控制机器狗进行足球对抗比赛。}
{1.通过ROS操作系统,调用CyberDog的传感器数据,实现简单的避障算法。\\
2.通过CyberDog的RGB相机和RealSense深度相机,利用YOLOv5模型定位足球位置。}
{1.实现了CyberDog的基本控制;2.实现了CyberDog的避障算法;3.实现了CyberDog的目标检测。4.能够进行简单的足球攻防。}

\datedsubsection{\textbf{基于机器学习的股票策略},人工智能导论课程项目(在做)}{2024.3 - 2024.6(预期)}
\Content
{针对A股市场,利用机器学习方法,建立交易策略。}
{建立设计神经网络模型,进行训练和预测。}
{通过PyTorch,建立了LSTM,MLP模型等,能够通过相关的交易特征数据给出一定的价格跟踪。}

\section{个人能力}
\datedsubsection{\textbf{数理基础}:已完成数学分析、高等代数、复变函数引论、常微分方程等课程学习,均取得A等级评定,具备良好的数学基础。}{}
\datedsubsection{\textbf{力学基础}:已完成动力学控制、连续介质力学、工程热力学等课程学习,具备力学视野。}{}
\datedsubsection{\textbf{代码基础}:有C/C++,Python语言基础,具有一定数据结构知识。对于Python中NumPy,Scikit-learn,PyTorch,OpenCV,Matplotlib等库均有使用经验,可以利用Python实现一些机器学习任务(利用SVM,Perceptron,KNN,CNN等不同算法完成过MNIST手写数字的分类问题)。}{}

\section{奖励荣誉}
\datedsubsection{\textbf{清华大学西南学子——学业优秀奖学金}}{2023}
\datedsubsection{\textbf{全国大学生数学建模比赛北京市二等奖}}{2023}
\end{document}